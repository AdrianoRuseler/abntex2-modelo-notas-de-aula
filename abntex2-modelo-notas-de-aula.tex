%% abtex2-modelo-artigo.tex, v-1.9.5 laurocesar
%% Copyright 2012-2015 by abnTeX2 group at http://www.abntex.net.br/ 
%%
%% This work may be distributed and/or modified under the
%% conditions of the LaTeX Project Public License, either version 1.3
%% of this license or (at your option) any later version.
%% The latest version of this license is in
%%   http://www.latex-project.org/lppl.txt
%% and version 1.3 or later is part of all distributions of LaTeX
%% version 2005/12/01 or later.
%%
%% This work has the LPPL maintenance status `maintained'.
%% 
%% The Current Maintainer of this work is the abnTeX2 team, led
%% by Lauro César Araujo. Further information are available on 
%% http://www.abntex.net.br/
%%
%% This work consists of the files abntex2-modelo-artigo.tex and
%% abntex2-modelo-references.bib
%%

% ------------------------------------------------------------------------
% ------------------------------------------------------------------------
% abnTeX2: Modelo de Artigo Acadêmico em conformidade com
% ABNT NBR 6022:2003: Informação e documentação - Artigo em publicação 
% periódica científica impressa - Apresentação
% ------------------------------------------------------------------------
% ------------------------------------------------------------------------

\documentclass[
	% -- opções da classe memoir --
	article,			% indica que é um artigo acadêmico
	11pt,				% tamanho da fonte
	oneside,			% para impressão apenas no verso. Oposto a twoside
	a4paper,			% tamanho do papel. 
	% -- opções da classe abntex2 --
	%chapter=TITLE,		% títulos de capítulos convertidos em letras maiúsculas
	%section=TITLE,		% títulos de seções convertidos em letras maiúsculas
	%subsection=TITLE,	% títulos de subseções convertidos em letras maiúsculas
	%subsubsection=TITLE % títulos de subsubseções convertidos em letras maiúsculas
	% -- opções do pacote babel --
	english,			% idioma adicional para hifenização
	brazil,				% o último idioma é o principal do documento
	sumario=tradicional
	]{abntex2-modelo-notas-de-aula}


% ---
% PACOTES
% ---

% ---
% Pacotes fundamentais 
% ---
\usepackage{lmodern}			% Usa a fonte Latin Modern
\usepackage[T1]{fontenc}		% Selecao de codigos de fonte.
\usepackage[utf8]{inputenc}		% Codificacao do documento (conversão automática dos acentos)
\usepackage{indentfirst}		% Indenta o primeiro parágrafo de cada seção.
\usepackage{nomencl} 			% Lista de simbolos
\usepackage{color}				% Controle das cores
\usepackage{graphicx}			% Inclusão de gráficos
\usepackage{microtype} 			% para melhorias de justificação
\usepackage{datetime}           % Tempo e hora.
\usepackage{epstopdf}
% ---
		
% ---
% Pacotes adicionais, usados apenas no âmbito do Modelo Canônico do abnteX2
% ---
\usepackage{lipsum}				% para geração de dummy text
% ---
		
% ---
% Pacotes de citações
% ---
\usepackage[brazilian,hyperpageref]{backref}	 % Paginas com as citações na bibl
\usepackage[alf]{abntex2cite}	% Citações padrão ABNT
% ---

\usepackage{listings}
\usepackage{amsmath}		
\usepackage{amssymb}
\usepackage{mathrsfs}
\usepackage{booktabs} % Para Tabelas
\usepackage{subfig}  % permite ter subfiguras
\usepackage{float}
\usepackage{tikz,pgfplots}
\usepackage{pdfpages}
\usepackage{longtable}


% Definição de cores
\definecolor{mygreen}{rgb}{0,0.6,0}
\definecolor{mygray}{rgb}{0.5,0.5,0.5}
\definecolor{mymauve}{rgb}{0.58,0,0.82}


\lstset{ %
	aboveskip=3mm,
	belowskip=3mm,
	backgroundcolor=\color{white},   % choose the background color; you must add \usepackage{color} or \usepackage{xcolor}
	basicstyle={\small\ttfamily},        % the size of the fonts that are used for the code
	breakatwhitespace=true,         % sets if automatic breaks should only happen at whitespace
	breaklines=true,                 % sets automatic line breaking
	captionpos=t,                    % sets the caption-position to bottom
	commentstyle=\color{mygreen},    % comment style
	columns=flexible,
	deletekeywords={...},            % if you want to delete keywords from the given language
	escapeinside={\%*}{*)},          % if you want to add LaTeX within your code
	extendedchars=true,              % lets you use non-ASCII characters; for 8-bits encodings only, does not work with UTF-8
	frame=tb,                        % adds a frame around the code
	keepspaces=true,                 % keeps spaces in text, useful for keeping indentation of code (possibly needs columns=flexible)
	keywordstyle=\color{blue},       % keyword style
	language=Matlab,                 % the language of the code
	morekeywords={*,...},            % if you want to add more keywords to the set
	numbers=none,                    % where to put the line-numbers; possible values are (none, left, right)
	numbersep=5pt,                   % how far the line-numbers are from the code
	numberstyle=\tiny\color{mygray}, % the style that is used for the line-numbers
	rulecolor=\color{black},         % if not set, the frame-color may be changed on line-breaks within not-black text (e.g. comments (green here))
	showspaces=false,                % show spaces everywhere adding particular underscores; it overrides 'showstringspaces'
	showstringspaces=false,          % underline spaces within strings only
	showtabs=false,                  % show tabs within strings adding particular underscores
	stepnumber=2,                    % the step between two line-numbers. If it's 1, each line will be numbered
	stringstyle=\color{mymauve},     % string literal style
	tabsize=3,                       % sets default tabsize to 3 spaces
	texcl=true,						 % Permite o uso de acentuação no código
	title=\lstname                   % show the filename of files included with \lstinputlisting; also try caption instead of title
}

%By default, listings does not support multi-byte encoding for source code. The extendedchar option only works for 8-bits encodings such as latin1.
%
%To handle UTF-8, you should tell listings how to interpret the special characters by defining them like so
\lstset{literate=	
	{á}{{\'a}}1 {é}{{\'e}}1 {í}{{\'i}}1 {ó}{{\'o}}1 {ú}{{\'u}}1
	{Á}{{\'A}}1 {É}{{\'E}}1 {Í}{{\'I}}1 {Ó}{{\'O}}1 {Ú}{{\'U}}1
	{à}{{\`a}}1 {è}{{\`e}}1 {ì}{{\`i}}1 {ò}{{\`o}}1 {ù}{{\`u}}1
	{À}{{\`A}}1 {È}{{\'E}}1 {Ì}{{\`I}}1 {Ò}{{\`O}}1 {Ù}{{\`U}}1
	{ä}{{\"a}}1 {ë}{{\"e}}1 {ï}{{\"i}}1 {ö}{{\"o}}1 {ü}{{\"u}}1
	{Ä}{{\"A}}1 {Ë}{{\"E}}1 {Ï}{{\"I}}1 {Ö}{{\"O}}1 {Ü}{{\"U}}1
	{â}{{\^a}}1 {ê}{{\^e}}1 {î}{{\^i}}1 {ô}{{\^o}}1 {û}{{\^u}}1
	{Â}{{\^A}}1 {Ê}{{\^E}}1 {Î}{{\^I}}1 {Ô}{{\^O}}1 {Û}{{\^U}}1	 
	{œ}{{\oe}}1 {Œ}{{\OE}}1 {æ}{{\ae}}1 {Æ}{{\AE}}1 {ß}{{\ss}}1
	{ű}{{\H{u}}}1 {Ű}{{\H{U}}}1 {ő}{{\H{o}}}1 {Ő}{{\H{O}}}1
	{ç}{{\c c}}1 {Ç}{{\c C}}1 {ø}{{\o}}1 {å}{{\r a}}1 {Å}{{\r A}}1
	{€}{{\EUR}}1 {£}{{\pounds}}1 {ã}{{\~a}}1 {õ}{{\~o}}1 {Ã}{{\~A}}1 {Õ}{{\~O}}1	
}

\renewcommand{\lstlistingname}{Código--fonte }% Listing -> Algorithm
\renewcommand{\lstlistlistingname}{Lista de códigos--fonte}% List of Listings -> List of Algorithms

% ---
% Configurações do pacote backref
% Usado sem a opção hyperpageref de backref
\renewcommand{\backrefpagesname}{Citado na(s) página(s):~}
% Texto padrão antes do número das páginas
\renewcommand{\backref}{}
% Define os textos da citação
\renewcommand*{\backrefalt}[4]{
	\ifcase #1 %
	Nenhuma citação no texto.%
	\or
	Citado na página #2.%
	\else
	Citado #1 vezes nas páginas #2.%
	\fi}%
% ---


% ---
% Informações de dados para CAPA e FOLHA DE ROSTO
% ---
\centro{Centro Tecnológico}
\departamento{Departamento de Engenharia Elétrica e Eletrônica}
\local{Florianópolis}
\data{\today}

\autor{Prof. Nome \textsc{Sobrenome}, Dr. Eng.}

\tipotrabalho{Notas de Aula}
\disciplina{Eletrônica de Potência I}
\codigo{EEL 7074} % Código da disciplina
\semestre{2015/2}
\aula{Aula 01}
\titulo{Modelo Canônico de Notas de Aula com \abnTeX}

\preambulo{Modelo canônico de notas de aula em conformidade
	com as normas ABNT apresentado à comunidade de usuários \LaTeX.}



% ---
% Configurações de aparência do PDF final

% alterando o aspecto da cor azul
\definecolor{blue}{RGB}{41,5,195}

% informações do PDF
\makeatletter
\hypersetup{
     	%pagebackref=true,
		pdftitle={\@title}, 
		pdfauthor={\@author},
    	pdfsubject={Modelo de artigo científico com abnTeX2},
	    pdfcreator={LaTeX with abnTeX2},
		pdfkeywords={abnt}{latex}{abntex}{abntex2}{atigo científico}, 
		colorlinks=true,       		% false: boxed links; true: colored links
    	linkcolor=blue,          	% color of internal links
    	citecolor=blue,        		% color of links to bibliography
    	filecolor=magenta,      		% color of file links
		urlcolor=blue,
		bookmarksdepth=4
}
\makeatother
% --- 

% ---
% compila o indice
% ---
\makeindex
% ---

% ---
% Altera as margens padrões
% ---
\setlrmarginsandblock{3cm}{3cm}{*}
\setulmarginsandblock{3cm}{3cm}{*}
\checkandfixthelayout
% ---


% --- 
% Espaçamentos entre linhas e parágrafos 
% --- 

% O tamanho do parágrafo é dado por:
\setlength{\parindent}{1.3cm}

% Controle do espaçamento entre um parágrafo e outro:
\setlength{\parskip}{0.2cm}  % tente também \onelineskip

% Espaçamento simples
\SingleSpacing

% ----
% Início do documento
% ----
\begin{document}

% Seleciona o idioma do documento (conforme pacotes do babel)
%\selectlanguage{english}
\selectlanguage{brazil}

% Retira espaço extra obsoleto entre as frases.
\frenchspacing 


\imprimircapaUFSC 

% ]  				% FIM DE ARTIGO EM DUAS COLUNAS
% ---


% ---
% inserir o sumario
% ---
\pdfbookmark[0]{\contentsname}{toc}
\tableofcontents*
\cleardoublepage
% ---

% ----------------------------------------------------------
% ELEMENTOS TEXTUAIS
% ----------------------------------------------------------
\textual
\pagestyle{notasUFSC}




% ----------------------------------------------------------
% Introdução
% ----------------------------------------------------------
\section*{Introdução}
\addcontentsline{toc}{section}{Introdução}

Este documento e seu código-fonte são exemplos de referência de uso da classe
\textsf{abntex2} e do pacote \textsf{abntex2cite}. O documento exemplifica a
elaboração de publicação periódica científica impressa produzida conforme a ABNT
NBR 6022:2003 \emph{Informação e documentação - Artigo em publicação periódica
científica impressa - Apresentação}.

A expressão ``Modelo canônico'' é utilizada para indicar que \abnTeX\ não é
modelo específico de nenhuma universidade ou instituição, mas que implementa tão
somente os requisitos das normas da ABNT. Uma lista completa das normas
observadas pelo \abnTeX\ é apresentada em \citeonline{abntex2classe}.

Sinta-se convidado a participar do projeto \abnTeX! Acesse o site do projeto em
\url{http://www.abntex.net.br/}. Também fique livre para conhecer,
estudar, alterar e redistribuir o trabalho do \abnTeX, desde que os arquivos
modificados tenham seus nomes alterados e que os créditos sejam dados aos
autores originais, nos termos da ``The \LaTeX\ Project Public
License''\footnote{\url{http://www.latex-project.org/lppl.txt}}.

Encorajamos que sejam realizadas customizações específicas deste documento.
Porém, recomendamos que ao invés de se alterar diretamente os arquivos do
\abnTeX, distribua-se arquivos com as respectivas customizações. Isso permite
que futuras versões do \abnTeX~não se tornem automaticamente incompatíveis com
as customizações promovidas. Consulte \citeonline{abntex2-wiki-como-customizar}
par mais informações.

Este exemplo deve ser utilizado como complemento do manual da classe
\textsf{abntex2} \cite{abntex2classe}, dos manuais do pacote
\textsf{abntex2cite} \cite{abntex2cite,abntex2cite-alf} e do manual da classe
\textsf{memoir} \cite{memoir}. Consulte o \citeonline{abntex2modelo} para obter
exemplos e informações adicionais de uso de \abnTeX\ e de \LaTeX.

% ----------------------------------------------------------
% Seção de explicações
% ----------------------------------------------------------
\newpage
\section{Exemplos de comandos}


\subsection{Inserindo código fonte}

\begin{lstlisting}[caption={Leitura dos dados simulados e conversão para estados topológicos.},label={lst:leituradadossim}]
% Pré definições iniciais
nsub=3;  % Numero de Sunmódulos
nbits=2*nsub; % Numero de bits necessários para representar os estados
nlevels=2*nsub+1; % Numero total de níveis

% Leitura dos pontos gerados por simulação
time=data(1,:)'; % extrai vetor de tempo
PWM=logical(data(2:end,:))'; % Conversão dos pulsos PWM para estados lógicos

% Cria vetor de string binário com os estados correspondentes
binstates=num2str([PWM(:,1) PWM(:,3) PWM(:,5) PWM(:,7) PWM(:,9) PWM(:,11)]);
state=fi(bin2dec(binstates),0,nbits,0); % Objeto numérico de ponto-fixo
\end{lstlisting}

\subsection{Figuras}



\begin{figure}[!h]
	\centering
	\includegraphics[width=1\linewidth]{figs/InversorTransformador}
	\caption{Conexão utilizada ao se empregar um transformador.}
	\label{fig:InversorTransformador}
\end{figure}


\subsection{Apresentando aquisições}


\begin{figure}[!h]
	\centering
	\begin{minipage}[t]{0.45\textwidth}	\centering
		\includegraphics[width=1\linewidth]{aqs/0002}
		\caption{Tensões nos capacitores de barramento e correntes de saída dos inversores da fase B ($W_{b1}$, $W_{b2}$ e $W_{b3}$) e da fase A ($W_{a1}$, $W_{a2}$ e $W_{a3}$).}
		\label{fig:0002}
	\end{minipage}
	\quad
	\begin{minipage}[t]{0.45\textwidth} \centering
		\includegraphics[width=1\linewidth]{aqs/TTZVSPWM0015}
		\caption{Tensões nos capacitores de barramento e correntes de saída dos inversores da fase C ($W_{c1}$, $W_{c2}$ e $W_{c3}$).}
		\label{fig:TTZVSPWM0015}
	\end{minipage}	
\end{figure}


\begin{figure}[!h]
	\centering
	\begin{minipage}[t]{0.45\textwidth}	\centering
		\includegraphics[width=1\linewidth]{aqs/tek0003}
		\caption{Tempo morto medido no braço 1 do inversor A3}
		\label{fig:tek0003}
	\end{minipage}
	\quad
	\begin{minipage}[t]{0.45\textwidth} 	\centering
		\includegraphics[width=1\linewidth]{aqs/tek0004}
		\caption{Tempo morto medido no braço 2 do inversor A3}
		\label{fig:tek0004}
	\end{minipage}	
\end{figure}


\begin{figure}[!h]
	\centering
	\begin{minipage}[t]{0.45\textwidth}	\centering
		\includegraphics[width=1\linewidth]{aqs/tek0001}
		\caption{Correntes na fase C com um degrau de carga.}
		\label{fig:currenttek0001}
	\end{minipage}
	\quad
	\begin{minipage}[t]{0.45\textwidth} 	\centering
		\includegraphics[width=1\linewidth]{aqs/tek0001FFT}
		\caption{Espectro das correntes na fase C com um degrau de carga.}
		\label{fig:currentFFTtek0001}
	\end{minipage}	
\end{figure}


\subsection{Margens}

A norma ABNT NBR 6022:2003 não estabelece uma margem específica a ser utilizada
no artigo científico. Dessa maneira, caso deseje alterar as margens, utilize os
comandos abaixo:

\begin{verbatim}
   \setlrmarginsandblock{3cm}{3cm}{*}
   \setulmarginsandblock{3cm}{3cm}{*}
   \checkandfixthelayout
\end{verbatim}

\subsection{Duas colunas}

É comum que artigos científicos sejam escritos em duas colunas. Para isso,
adicione a opção \texttt{twocolumn} à classe do documento, como no exemplo:

\begin{verbatim}
   \documentclass[article,11pt,oneside,a4paper,twocolumn]{abntex2}
\end{verbatim}

É possível indicar pontos do texto que se deseja manter em apenas uma coluna,
geralmente o título e os resumos. Os resumos em única coluna em documentos com
a opção \texttt{twocolumn} devem ser escritos no ambiente
\texttt{resumoumacoluna}:

\begin{verbatim}
   \twocolumn[              % INICIO DE ARTIGO EM DUAS COLUNAS

     \maketitle             % pagina de titulo

     \renewcommand{\resumoname}{Nome do resumo}
     \begin{resumoumacoluna}
        Texto do resumo.
      
        \vspace{\onelineskip}
 
        \noindent
        \textbf{Palavras-chave}: latex. abntex. editoração de texto.
     \end{resumoumacoluna}
   
   ]                        % FIM DE ARTIGO EM DUAS COLUNAS
\end{verbatim}

\subsection{Recuo do ambiente \texttt{citacao}}

Na produção de artigos (opção \texttt{article}), pode ser útil alterar o recuo
do ambiente \texttt{citacao}. Nesse caso, utilize o comando:

\begin{verbatim}
   \setlength{\ABNTEXcitacaorecuo}{1.8cm}
\end{verbatim}

Quando um documento é produzido com a opção \texttt{twocolumn}, a classe
\textsf{abntex2} automaticamente altera o recuo padrão de 4 cm, definido pela
ABNT NBR 10520:2002 seção 5.3, para 1.8 cm.

\section{Cabeçalhos e rodapés customizados}

Diferentes estilos de cabeçalhos e rodapés podem ser criados usando os
recursos padrões do \textsf{memoir}.

Um estilo próprio de cabeçalhos e rodapés pode ser diferente para páginas pares
e ímpares. Observe que a diferenciação entre páginas pares e ímpares só é
utilizada se a opção \texttt{twoside} da classe \textsf{abntex2} for utilizado.
Caso contrário, apenas o cabeçalho padrão da página par (\emph{even}) é usado.

Veja o exemplo abaixo cria um estilo chamado \texttt{meuestilo}. O código deve
ser inserido no preâmbulo do documento.

\begin{verbatim}
%%criar um novo estilo de cabeçalhos e rodapés
\makepagestyle{meuestilo}
  %%cabeçalhos
  \makeevenhead{meuestilo} %%pagina par
     {topo par à esquerda}
     {centro \thepage}
     {direita}
  \makeoddhead{meuestilo} %%pagina ímpar ou com oneside
     {topo ímpar/oneside à esquerda}
     {centro\thepage}
     {direita}
  \makeheadrule{meuestilo}{\textwidth}{\normalrulethickness} %linha
  %% rodapé
  \makeevenfoot{meuestilo}
     {rodapé par à esquerda} %%pagina par
     {centro \thepage}
     {direita} 
  \makeoddfoot{meuestilo} %%pagina ímpar ou com oneside
     {rodapé ímpar/onside à esquerda}
     {centro \thepage}
     {direita}
\end{verbatim}

Para usar o estilo criado, use o comando abaixo imediatamente após um dos
comandos de divisão do documento. Por exemplo:

\begin{verbatim}
   \begin{document}
     %%usar o estilo criado na primeira página do artigo:
     \pretextual
     \pagestyle{meuestilo}
     
     \maketitle
     ...
     
     %%usar o estilo criado nas páginas textuais
     \textual
     \pagestyle{meuestilo}
     
     \chapter{Novo capítulo}
     ...
   \end{document}  
\end{verbatim}
   
Outras informações sobre cabeçalhos e rodapés estão disponíveis na seção 7.3 do
manual do \textsf{memoir} \cite{memoir}.

\section{Mais exemplos no Modelo Canônico de Trabalhos Acadêmicos}

Este modelo de artigo é limitado em número de exemplos de comandos, pois são
apresentados exclusivamente comandos diretamente relacionados com a produção de
artigos.

Para exemplos adicionais de \abnTeX\ e \LaTeX, como inclusão de figuras,
fórmulas matemáticas, citações, e outros, consulte o documento
\citeonline{abntex2modelo}.

\section{Consulte o manual da classe \textsf{abntex2}}

Consulte o manual da classe \textsf{abntex2} \cite{abntex2classe} para uma
referência completa das macros e ambientes disponíveis.

% ---
% Finaliza a parte no bookmark do PDF, para que se inicie o bookmark na raiz
% ---
\bookmarksetup{startatroot}% 
% ---

% ---
% Conclusão
% ---
\section*{Considerações finais}
\addcontentsline{toc}{section}{Considerações finais}

\lipsum[1]

\begin{citacao}
\lipsum[2]
\end{citacao}

\lipsum[3]

% ----------------------------------------------------------
% ELEMENTOS PÓS-TEXTUAIS
% ----------------------------------------------------------
\postextual
\pagestyle{plain}

% ---
% Título e resumo em língua estrangeira
% ---

% \twocolumn[    		% INICIO DE ARTIGO EM DUAS COLUNAS



% ]  				% FIM DE ARTIGO EM DUAS COLUNAS
% ---

\newpage
\pagestyle{notasUFSC}
% ----------------------------------------------------------
% Referências bibliográficas
% ----------------------------------------------------------
\bibliography{abntex2-modelo-references}

% ----------------------------------------------------------
% Glossário
% ----------------------------------------------------------
%
% Há diversas soluções prontas para glossário em LaTeX. 
% Consulte o manual do abnTeX2 para obter sugestões.
%
%\glossary

% ----------------------------------------------------------
% Apêndices
% ----------------------------------------------------------

\newpage

% ---
% Inicia os apêndices
% ---
\begin{apendicesenv}

% ----------------------------------------------------------
\chapter{Nullam elementum urna vel imperdiet sodales elit ipsum pharetra ligula
ac pretium ante justo a nulla curabitur tristique arcu eu metus}
% ----------------------------------------------------------
\lipsum[55-57]

\end{apendicesenv}
% ---

\newpage
% ----------------------------------------------------------
% Anexos
% ----------------------------------------------------------
\cftinserthook{toc}{AAA}
% ---
% Inicia os anexos
% ---
%\anexos
\begin{anexosenv}

\chapter{Datasheet para o conjunto inversor monofásico SPCIM 450-60-20}

\includepdf[pages=-]{docs/SPCIM450-60-20.pdf}

\end{anexosenv}

\end{document}
